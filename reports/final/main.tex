\documentclass[12pt,a4paper]{mwart}

% =============================================
% PACKAGES
% =============================================
\usepackage[utf8]{inputenc}
\usepackage[T1]{fontenc}
\usepackage[english,polish]{babel}
\usepackage{polski}
\RequirePackage[
    a4paper,
    left=2.5cm,
    right=2.5cm,
    top=2.5cm,
    bottom=2.5cm,
    marginparwidth=2.5cm
]{geometry}
\usepackage{graphicx}
\usepackage{placeins}
\usepackage{amsmath, amssymb}
\usepackage{booktabs}
\usepackage{hyperref}
\usepackage{enumitem}
\usepackage{float}
\usepackage{caption}
\usepackage{subcaption}
\usepackage{mlmodern}
\usepackage[export]{adjustbox}

\usepackage{graphicx} 
\usepackage{float}    

\RequirePackage[
    disable,
    colorinlistoftodos,
    textsize=tiny
]{todonotes}
\RequirePackage{marginnote}
\let\marginpar\marginnote

\hypersetup{
    colorlinks=false
}

% =============================================
% DOCUMENT
% =============================================
\begin{document}

% =============================================
% TITLE PAGE
% =============================================
\begin{titlepage}
    \centering
    
    \vspace*{2cm}
    
    {\LARGE \textbf{Dokumentacja końcowa}}\\[1.5cm]
    
    {\Huge \textbf{System wspierający planowanie zapasów kawy}}\\[2cm]
    
    % --- Autorzy ---
    {\large 
    \textbf{Zespół 2:}\\[0.3cm]
    Krzysztof Fijałkowski\\
    Jakub Kordel\\
    Tomasz Owienko\\
    Tomasz Świderski
    }\\[1cm]
    
    % --- Zespół i wersja ---
    {\large
   
    \textbf{Wersja dokumentu: 2.0}\\[0.3cm]
    21.01.2026
    }
    
    \vfill
\end{titlepage}

% =============================================
% TABLE OF CONTENTS
% =============================================
{
\tableofcontents
\clearpage
}

% =============================================
% INTRODUCTION
% =============================================
\section{Wprowadzenie}

\subsection{Problem biznesowy}
W dużych organizacjach posiadających wiele biurowców, zarządzanie zapasami kawy staje się skomplikowanym procesem logistycznym. Każdy biurowiec ma inną specyfikę zużycia, pojemność magazynową oraz lokalizację. 
Rynek dostawców kawy jest zróżnicowany -- poszczególni dystrybutorzy oferują różne ceny bazowe, różne systemy rabatowe zależne od wielkości zamówienia oraz różne czasy i koszty dostaw do poszczególnych lokalizacji.

Kluczowym wyzwaniem jest zapewnienie ciągłości dostaw we wszystkich biurowcach przy jednoczesnej minimalizacji całkowitych kosztów, obejmujących zakup kawy (z uwzględnieniem rabatów) oraz koszty logistyczne. 
Decyzje zakupowe muszą uwzględniać nie tylko bieżące zapotrzebowanie, ale także dostępne zapasy, straty magazynowe (np. utrata świeżości) oraz ograniczenia podażowe po stronie dystrybutorów.

System ma za zadanie automatyzować proces decyzyjny, generując optymalny harmonogram zamówień dla całej sieci biurowców na podstawie prognozowanego zapotrzebowania, aktualnych cenników i ograniczeń logistycznych. System jest przeznaczony dla centralnego operatora koordynującego zakupy dla grupy biurowców mającego dostęp do wielu dystrybutorów kawy.

\subsection{Założenia modelu}

Przyjęto następujące założenia dla problemu optymalizacji dostaw kawy w wersji rozszerzonej:

\begin{itemize}
    \item \textbf{Wielu dystrybutorów i wiele biurowców:} System obsługuje wiele biurowców zaopatrywanych przez wielu niezależnych dystrybutorów.
    \item \textbf{Zróżnicowane koszty transportu i czasy dostaw:} Koszt stały dostawy oraz czas oczekiwania na zamówienie zależą od pary (dystrybutor, biurowiec).
    \item \textbf{System rabatowy:} Dystrybutorzy oferują ceny zależne od wielkości zamówienia (progi ilościowe). Ceny są progresywne - przekroczenie progu ilościowego skutkuje niższą ceną jednostkową dla części zamówienia będącej powyżej tego progu.
    \item \textbf{Ograniczenia podażowe:} Dystrybutorzy mają skończone zapasy danych dniach.
    \item \textbf{Zamówienia historyczne:} Model uwzględnia zamówienia złożone w przeszłości, które są w trakcie realizacji (już złożone, ale jeszcze nie dotarły do magazynu), co wpływa na bilans zapasów w przyszłości.
    \item \textbf{Straty magazynowe:} Uwzględniono współczynnik dziennej utraty kawy (np. zużycie, utrata świeżości, straty proceduralne), wyrażony jako procent bieżącego stanu magazynowego.
    \item \textbf{Horyzont planowania:} Decyzje podejmowane są dla skończonego horyzontu czasowego (np. 7 dni), przy czym model musi zapewnić zaspokojenie popytu w każdym dniu tego okresu.
    \item \textbf{Pojemność magazynów:} Każdy biurowiec posiada ograniczoną pojemność magazynową, której nie można przekroczyć.
\end{itemize}

% =============================================
% MATHEMATICAL MODEL
% =============================================
\section{Model matematyczny w wersji zaawansowanej}

\subsection{Zbiory}
\begin{itemize}
    \item \(T = \{1,2,\dots,7\}\) – zbiór analizowanych dni,
    \item \(\hat{T} = \{\hat{T}_{\max}, \dots, 0\}\) – zbiór dni historycznych przed rozpoczęciem planowania,
    \item \(D\) – zbiór dystrybutorów, indeksowany przez \(d\),
    \item \(B\) – zbiór biurowców, indeksowany przez \(b\),
    \item \(L = \{1,2,\dots,L_{max}\}\) – zbiór progów rabatowych, indeksowany przez \(l\).
\end{itemize}

\subsection{Parametry}
\begin{itemize}
    \item \(V^{max}_{b}\) – maksymalna pojemność magazynu w biurowcu \(b\) [kg],
    \item \(Q_{l}\) – dolny próg ilościowy rabatu \(l\) [kg],
    \item \(P_{d,t,0}\) – cena jednostkowa od dystrybutora \(d\) w dniu \(t\) pod progiem \(1\) [zł/kg],
    \item \(P_{d,t,l}\) – cena jednostkowa od dystrybutora \(d\) w dniu \(t\) nad progiem \(l\) [zł/kg],
    \item \(C^{fix}_{d,b}\) – stały koszt realizacji dostawy od dystrybutora \(d\) do biurowca \(b\) [zł],
    \item \(D_{b,t}\) – zapotrzebowanie na kawę w biurowcu \(b\) w dniu \(t\) [kg],
    \item \(I_{b,0}\) – początkowy stan magazynu kawy w biurowcu \(b\) [kg],
    \item \(\alpha\) – procent dziennej utraty kawy,
    \item \(S_{d,t}\) – maksymalna dostępna ilość kawy u dystrybutora \(d\) w dniu \(t\) [kg],
    \item \(S^{\max} = \max_{(d,t)\in D \times T} S_{d,t}\) – maksymalna dzienna dostępność,
    \item \(X_{d,b}\) – czas dostawy kawy od dystrybutora \(d\) do biurowca \(b\) [dni],
    \item \(x^0_{d,b,\tau,l} \ge 0\) – zamówienia historyczne w dniu \(\tau \in \hat{T}\), które już zostały złożone i są stałe.
\end{itemize}

\subsection{Zmienne decyzyjne}
\begin{itemize}
    \item \(x_{d,b,t,0} \ge 0\) – ilość kawy zamówiona od dystrybutora \(d\) do biurowca \(b\) w dniu \(t \in T\) przed progiem rabatowym \(1\) [kg],
    \item \(x_{d,b,t,l} \ge 0\) – ilość kawy zamówiona od dystrybutora \(d\) do biurowca \(b\) w dniu \(t \in T\) nad progiem rabatowym \(l\) [kg],
    \item \(I_{b,t} \ge 0\) – stan magazynu kawy w biurowcu \(b\) na koniec dnia \(t \in T\) [kg],
    \item \(y^{skl}_{d,b,t} \in \{0,1\}\) – czy w dniu \(t \in T\) składane jest zamówienie,
    \item \(y_{d,b,t,l} \in \{0,1\}\) – zmienna binarna przekroczenia progu rabatowego \(l\) w dniu \(t \in T\), w zamówieniu biurowca \(b \in B\) od dystrybutora \(d \in D\).
\end{itemize}

\subsection{Funkcja celu}
\begin{equation}
\min \sum_{t \in T} \sum_{b \in B} \sum_{d \in D}
P_{d,t,0} \, x_{d,b,t,0} + \sum_{t \in T} \sum_{b \in B} \sum_{d \in D} \sum_{l \in L} 
P_{d,t,l} \, x_{d,b,t,l} 
+ \sum_{t \in T} \sum_{b \in B} \sum_{d \in D} 
C^{fix}_{d,b}\, y^{skl}_{d,b,t}
\end{equation}

\subsection{Ograniczenia}

\paragraph{Bilans zapasów z uwzględnieniem zamówień historycznych}
\begin{align}
I_{b,t} = (1-\alpha) I_{b,t-1} + \sum_{d \in D}\sum_{\substack{\tau \in T \\ \tau + X_{d,b} = t}} x_{d,b,\tau,0} 
+ \sum_{d \in D} \sum_{l \in L}
\left(
\sum_{\substack{\tau \in T \\ \tau + X_{d,b} = t}} x_{d,b,\tau,l}
+
\sum_{\substack{\tau \in \hat{T} \\ \tau + X_{d,b} = t}} x^0_{d,b,\tau,l}
\right)
- D_{b,t}, \\
\forall b \in B, \; t \in T
\end{align}

\paragraph{Pojemność magazynu}
\begin{equation}
I_{b,t} \le V^{max}_{b}, 
\quad \forall b \in B, \; t \in T
\end{equation}

\paragraph{Powiązanie zamówienia ze zmienną binarną}
\begin{equation}
x_{d,b,t,0} \le S_{d,t} \, y^{skl}_{d,b,t}, 
\quad \forall d \in D, \; b \in B, \; t \in T
\end{equation}

\paragraph{Ograniczenie dostępności u dystrybutora}
\begin{equation}
\sum_{b \in B} x_{d,b,t,0} + \sum_{b \in B} \sum_{l \in L} x_{d,b,t,l} \le S_{d,t}, 
\quad \forall d \in D, \; t \in T
\end{equation}

\paragraph{Ograniczenia progowe}
\begin{align}
0 \le x_{d,b,t,0} &\le Q_1 \, && \forall d \in D, b \in B, t \in T\\
0 \le x_{d,b,t,l} &\le (Q_{l+1} - Q_{l}) \, y_{d,b,t,l}, && \forall d \in D, b \in B, t \in T, l=1,\dots,L_{max}-1\\
0 \le x_{d,b,t,1} &\le S^{max} \, y_{d,b,t,l}, && \forall d \in D, b \in B, t \in T, l=L_{max}\\
x_{d,b,t,0} &\ge Q_1 \, y_{d,b,t,1}, && \forall d \in D, b \in B, t \in T \\
x_{d,b,t,l} &\ge (Q_{l+1} - Q_{l}) \, y_{d,b,t,l+1}, && \forall d \in D, b \in B, t \in T, l=1,\dots,L_{max} - 1 \\
\end{align}

\paragraph{Warunki nieujemności i binarności}
\begin{align}
x_{d,b,t,l} &\ge 0, && \forall d \in D, \; b \in B, \; t \in T, \; l \in L \\
I_{b,t} &\ge 0, && \forall b \in B, \; t \in T \\
y^{skl}_{d,b,t} &\in \{0,1\}, && \forall d \in D, \; b \in B, \; t \in T \\
y^{rab}_{d,b,t,l} &\in \{0,1\}, && \forall d \in D, \; b \in B, \; t \in T, \; l \in L
\end{align}

% =============================================
% APPLICATION
% =============================================
\section{Aplikacja}

(Sekcja w przygotowaniu - miejsce przeznaczone na opis interfejsu użytkownika i funkcjonalności aplikacji webowej).

\vspace{10cm}
\begin{center}
    \textit{[Miejsce na zrzuty ekranu aplikacji]}
\end{center}

% =============================================
% TESTS
% =============================================
\section{Testy modelu}

W celu weryfikacji poprawności działania modelu oraz jego odporności na skrajne przypadki, przeprowadzono szereg testów scenariuszowych. Poniżej omówiono wybrane, najbardziej interesujące przypadki testowe. Plik \texttt{.ipynb} zawierający komplet scenariuszy testowych znajduje się w załączniku.

\subsection{Scenariusz podstawowy (Wielu dystrybutorów i biurowców)}
Scenariusz ten weryfikuje podstawową poprawność działania modelu w środowisku z 2 dystrybutorami i 2 biurowcami. Dystrybutorzy różnią się cenami oraz czasem dostawy, a biurowce mają różne zapotrzebowanie.

\textbf{Parametry testu:}
\begin{itemize}
    \item Horyzont czasu: \(T=7\) dni.
    \item 2 Dystrybutorów, 2 Biurowce, 2 Progi rabatowe.
    \item Pojemność magazynów: \(V^{max} = [100, 150]\) kg.
    \item Progi ilościowe: \(Q = [0, 30, 60]\) kg.
    \item Ceny bazowe (poniżej 30 kg): 
    \begin{itemize}
        \item Dystrybutor 1 (D0): 12.00 zł/kg (droższy bazowo), Czas dostawy: 1 dzień (B0), 2 dni (B1).
        \item Dystrybutor 2 (D1): 11.00 zł/kg (tańszy bazowo), Czas dostawy: 2 dni (B0), 1 dzień (B1).
    \end{itemize}
    \item Ceny rabatowe:
    \begin{itemize}
        \item D0: 10.00 zł (30-60kg), 8.00 zł (>60kg).
        \item D1: 9.50 zł (30-60kg), 7.50 zł (>60kg).
    \end{itemize}
    \item Koszty stałe dostawy:
    \begin{itemize}
        \item D0: 50 zł (do B0), 60 zł (do B1).
        \item D1: 45 zł (do B0), 55 zł (do B1).
    \end{itemize}
    \item Zapotrzebowanie:
    \begin{itemize}
        \item B0: ok. 10-15 kg dziennie.
        \item B1: ok. 15-20 kg dziennie.
    \end{itemize}
\end{itemize}

\begin{table}[H]
    \tiny
    \begin{tabular}{lrrrlrrrr}
\toprule
Building & Day & Demand (kg) & Ordered (kg) & Orders From & Inventory (kg) & Purchase Cost (PLN) & Fixed Cost (PLN) & Total Cost (PLN) \\
\midrule
B1 & 1 & 10.000000 & 83.660000 & D1, D2 & 8.000000 & 814.070000 & 95.000000 & 909.070000 \\
B1 & 2 & 12.000000 & 0.000000 & None & -0.000000 & 0.000000 & 0.000000 & 0.000000 \\
B1 & 3 & 15.000000 & 0.000000 & None & 63.860000 & 0.000000 & 0.000000 & 0.000000 \\
B1 & 4 & 10.000000 & 0.000000 & None & 47.480000 & 0.000000 & 0.000000 & 0.000000 \\
B1 & 5 & 11.000000 & 0.000000 & None & 31.730000 & -0.000000 & 0.000000 & -0.000000 \\
B1 & 6 & 13.000000 & 0.000000 & None & 15.560000 & -0.000000 & 0.000000 & -0.000000 \\
B1 & 7 & 14.000000 & 0.000000 & None & 0.000000 & 0.000000 & 0.000000 & 0.000000 \\
B2 & 1 & 15.000000 & 7.200000 & D2 & 12.000000 & 79.200000 & 55.000000 & 134.200000 \\
B2 & 2 & 18.000000 & 108.390000 & D2 & -0.000000 & 977.950000 & 55.000000 & 1032.950000 \\
B2 & 3 & 20.000000 & 0.000000 & None & 88.390000 & 0.000000 & 0.000000 & 0.000000 \\
B2 & 4 & 15.000000 & 0.000000 & None & 64.550000 & 0.000000 & 0.000000 & 0.000000 \\
B2 & 5 & 16.000000 & 0.000000 & None & 42.100000 & 0.000000 & 0.000000 & 0.000000 \\
B2 & 6 & 19.000000 & 0.000000 & None & 18.890000 & 0.000000 & 0.000000 & 0.000000 \\
B2 & 7 & 17.000000 & 0.000000 & None & 0.000000 & 0.000000 & 0.000000 & 0.000000 \\
\bottomrule
\end{tabular}
\end{table}

\begin{figure}[H]
    \centering
    \includegraphics[width=\textwidth]{img/test_1_1.png}
    \includegraphics[width=\textwidth]{img/test_1_2.png}
    \includegraphics[width=\textwidth]{img/test_1_3.png}
\end{figure}

\textbf{Wynik:} Model poprawnie zidentyfikował tańszego dostawcę dla każdego biurowca, uwzględniając koszty stałe. Zauważono cykliczność zamówień (co 2-3 dni) w celu przekroczenia progów rabatowych i zminimalizowania liczby opłat za transport, przy jednoczesnym unikaniu nadmiernego magazynowania (koszt \(\alpha=0.1\)).

\subsection{Scenariusz: Konkurencja cenowa i rabaty ilościowe}
W tym przypadku jeden z dystrybutorów oferuje znacznie korzystniejsze rabaty przy dużych zamówieniach, podczas gdy drugi ma niższe ceny dla małych zamówień.

\textbf{Parametry testu:}
\begin{itemize}
    \item 1 Biurowiec, 2 Dystrybutorów.
    \item Dystrybutor 1 (D0): Cena bazowa 15 zł, ale powyżej 80kg cena spada do 7 zł! Koszt stały 60 zł.
    \item Dystrybutor 2 (D1): Cena bazowa 10 zł, ale powyżej 80kg cena to 9 zł. Koszt stały 40 zł.
    \item Popyt: Przez większość dni 20 kg, ale w dniu 4. nagły skok do 200 kg.
\end{itemize}

\begin{table}[H]
    \tiny
\begin{tabular}{lrrrlrrrr}
\toprule
Building & Day & Demand (kg) & Ordered (kg) & Orders From & Inventory (kg) & Purchase Cost (PLN) & Fixed Cost (PLN) & Total Cost (PLN) \\
\midrule
B1 & 1 & 20.000000 & 53.710000 & D2 & 43.210000 & 530.280000 & 40.000000 & 570.280000 \\
B1 & 2 & 20.000000 & 0.000000 & None & 21.050000 & 0.000000 & 0.000000 & 0.000000 \\
B1 & 3 & 20.000000 & 0.000000 & None & 0.000000 & 0.000000 & 0.000000 & 0.000000 \\
B1 & 4 & 200.000000 & 266.540000 & D1 & 66.540000 & 2385.780000 & 60.000000 & 2445.780000 \\
B1 & 5 & 20.000000 & 0.000000 & None & 43.210000 & 0.000000 & 0.000000 & 0.000000 \\
B1 & 6 & 20.000000 & 0.000000 & None & 21.050000 & 0.000000 & 0.000000 & 0.000000 \\
B1 & 7 & 20.000000 & 0.000000 & None & 0.000000 & 0.000000 & 0.000000 & 0.000000 \\
\bottomrule
\end{tabular}
\end{table}

\begin{figure}[H]
    \centering
    \includegraphics[width=\textwidth]{img/test_2_1.png}
    \includegraphics[width=\textwidth]{img/test_2_2.png}
\end{figure}

\textbf{Wynik:} Model w dniach o niskim popycie korzystał z Dystrybutora 2 (małe, tanie zamówienia). Jednak w obliczu dużego zapotrzebowania w dniu 4., system przełączył się na Dystrybutora 1, zamawiając dużą partię towaru, aby skorzystać z agresywnego rabatu (7 zł/kg), co zrekompensowało wyższy koszt stały i bazowy.

\subsection{Scenariusz: Arbitraż cenowy w czasie}
Scenariusz zakłada drastyczną zmianę cen w czasie.

\textbf{Parametry testu:}
\begin{itemize}
    \item 1 Dystrybutor, 1 Biurowiec.
    \item Ceny w dniach 1-2 oraz 4-7 wynoszą ok. 13-14 zł/kg.
    \item W dniu 3. (``Black Friday'') cena spada do 5 zł/kg (4 zł/kg w hurcie).
    \item Popyt stały: 20 kg/dzień.
    \item Magazyn: duży (200 kg).
\end{itemize}

\begin{table}[H]
    \tiny
\begin{tabular}{lrrrlrrrr}
\toprule
Building & Day & Demand (kg) & Ordered (kg) & Orders From & Inventory (kg) & Purchase Cost (PLN) & Fixed Cost (PLN) & Total Cost (PLN) \\
\midrule
B1 & 1 & 20.000000 & 32.540000 & D1 & 21.740000 & 488.090000 & 60.000000 & 548.090000 \\
B1 & 2 & 20.000000 & 0.000000 & None & 0.000000 & 0.000000 & 0.000000 & 0.000000 \\
B1 & 3 & 20.000000 & 118.970000 & D1 & 98.970000 & 595.880000 & 60.000000 & 655.880000 \\
B1 & 4 & 20.000000 & 0.000000 & None & 71.050000 & 0.000000 & 0.000000 & 0.000000 \\
B1 & 5 & 20.000000 & 0.000000 & None & 45.370000 & -0.000000 & 0.000000 & -0.000000 \\
B1 & 6 & 20.000000 & 0.000000 & None & 21.740000 & 0.000000 & 0.000000 & 0.000000 \\
B1 & 7 & 20.000000 & 0.000000 & None & 0.000000 & 0.000000 & 0.000000 & 0.000000 \\
\bottomrule
\end{tabular}
\end{table}

\begin{figure}[H]
    \centering
    \includegraphics[width=\textwidth]{img/test_3_1.png}
    \includegraphics[width=\textwidth]{img/test_3_2.png}
\end{figure}

\textbf{Wynik:} System złożył minimalne zamówienia na początku tygodnia, a w dniu 3. dokonał ogromnego zakupu (bliskiego limitu magazynowego), ``tankując'' tanią kawę na zapas na resztę tygodnia. Oszczędność na cenie zakupu znacznie przewyższyła koszty psucia się kawy (\(\alpha=0.08\)).

\subsection{Scenariusz: Wyścig o zasoby (Ograniczona podaż)}
Test weryfikuje zachowanie systemu przy deficycie towaru u dostawcy.

\textbf{Parametry testu:}
\begin{itemize}
    \item 3 Biurowce, 1 Dystrybutor.
    \item Łączny popyt biurowców: \(25+30+20 = 75\) kg/dzień.
    \item Limit dostaw u dystrybutora (\(S\)): 100 kg/dzień.
    \item Jest to limit bliski łącznego popytu, co utrudnia robienie zapasów przez wszystkie biurowce na raz.
\end{itemize}

\begin{table}[H]
    \tiny
\begin{tabular}{lrrrlrrrr}
\toprule
Building & Day & Demand (kg) & Ordered (kg) & Orders From & Inventory (kg) & Purchase Cost (PLN) & Fixed Cost (PLN) & Total Cost (PLN) \\
\midrule
B1 & 1 & 25.000000 & 20.250000 & D1 & -0.000000 & 202.500000 & 40.000000 & 242.500000 \\
B1 & 2 & 25.000000 & 64.100000 & D1 & 39.100000 & 582.820000 & 40.000000 & 622.820000 \\
B1 & 3 & 25.000000 & 0.000000 & None & 12.150000 & 0.000000 & 0.000000 & 0.000000 \\
B1 & 4 & 25.000000 & 13.460000 & D1 & 0.000000 & 134.600000 & 40.000000 & 174.600000 \\
B1 & 5 & 25.000000 & 79.020000 & D1 & 54.020000 & 702.130000 & 40.000000 & 742.130000 \\
B1 & 6 & 25.000000 & 0.000000 & None & 26.320000 & 0.000000 & 0.000000 & 0.000000 \\
B1 & 7 & 25.000000 & 0.000000 & None & 0.000000 & 0.000000 & 0.000000 & 0.000000 \\
B2 & 1 & 30.000000 & 56.830000 & D1 & 31.580000 & 524.630000 & 50.000000 & 574.630000 \\
B2 & 2 & 30.000000 & 0.000000 & None & 0.000000 & 0.000000 & 0.000000 & 0.000000 \\
B2 & 3 & 30.000000 & 94.820000 & D1 & 64.820000 & 828.560000 & 50.000000 & 878.560000 \\
B2 & 4 & 30.000000 & 0.000000 & None & 31.580000 & 0.000000 & 0.000000 & 0.000000 \\
B2 & 5 & 30.000000 & 0.000000 & None & 0.000000 & 0.000000 & 0.000000 & 0.000000 \\
B2 & 6 & 30.000000 & 61.580000 & D1 & 31.580000 & 562.630000 & 50.000000 & 612.630000 \\
B2 & 7 & 30.000000 & 0.000000 & None & 0.000000 & 0.000000 & 0.000000 & 0.000000 \\
B3 & 1 & 20.000000 & 20.680000 & D1 & 5.430000 & 206.770000 & 60.000000 & 266.770000 \\
B3 & 2 & 20.000000 & 35.900000 & D1 & 21.050000 & 357.180000 & 60.000000 & 417.180000 \\
B3 & 3 & 20.000000 & 0.000000 & None & 0.000000 & 0.000000 & 0.000000 & 0.000000 \\
B3 & 4 & 20.000000 & 86.540000 & D1 & 66.540000 & 762.320000 & 60.000000 & 822.320000 \\
B3 & 5 & 20.000000 & 0.000000 & None & 43.210000 & 0.000000 & 0.000000 & 0.000000 \\
B3 & 6 & 20.000000 & 0.000000 & None & 21.050000 & 0.000000 & 0.000000 & 0.000000 \\
B3 & 7 & 20.000000 & 0.000000 & None & 0.000000 & 0.000000 & 0.000000 & 0.000000 \\
\bottomrule
\end{tabular}
\end{table}

\begin{figure}[H]
    \centering
    \includegraphics[width=\textwidth]{img/test_4_1.png}
    \includegraphics[width=\textwidth]{img/test_4_2.png}
    \includegraphics[width=\textwidth]{img/test_4_3.png}
    \includegraphics[width=\textwidth]{img/test_4_4.png}
\end{figure}

\textbf{Wynik:} Model musiał ``wygładzić'' zamówienia w czasie. Biurowce nie mogły zamawiać dużych partii jednocześnie (by skorzystać z rabatów), ponieważ przekroczyłyby limit dystrybutora. Harmonogram zamówień został rozsunięty w czasie -- biurowce zamawiały naprzemiennie większe partie, lub zadowalały się mniejszymi dostawami codziennymi, aby zmieścić się w limicie podaży.

\subsection{Scenariusz: Zamówienia historyczne (Dostawy w toku)}
Sprawdzenie obsługi zamówień będących w transporcie.

\textbf{Parametry testu:}
\begin{itemize}
    \item 2 Biurowce, 2 Dystrybutorów.
    \item Czas dostawy: 1-2 dni.
    \item Stan początkowy:
    \begin{itemize}
        \item Zamówienie do B0 złożone 2 dni temu (wielkość 25 kg) -- dociera w dniu 1.
        \item Zamówienie do B1 złożone 1 dzień temu (wielkość 35 kg) -- dociera w dniu 1.
    \end{itemize}
    \item Popyt dzienny: ok. 15 kg.
\end{itemize}

\begin{table}[H]
    \tiny
\begin{tabular}{lrrrlrrrr}
\toprule
Building & Day & Demand (kg) & Ordered (kg) & Orders From & Inventory (kg) & Purchase Cost (PLN) & Fixed Cost (PLN) & Total Cost (PLN) \\
\midrule
B1 & 1 & 15.000000 & 0.000000 & None & 19.000000 & 0.000000 & 0.000000 & 0.000000 \\
B1 & 2 & 15.000000 & 91.730000 & D2 & 2.100000 & 822.140000 & 50.000000 & 872.140000 \\
B1 & 3 & 15.000000 & 0.000000 & None & 78.620000 & 0.000000 & 0.000000 & 0.000000 \\
B1 & 4 & 15.000000 & 0.000000 & None & 55.760000 & 0.000000 & 0.000000 & 0.000000 \\
B1 & 5 & 15.000000 & 0.000000 & None & 35.190000 & -0.000000 & 0.000000 & -0.000000 \\
B1 & 6 & 15.000000 & 0.000000 & None & 16.670000 & 0.000000 & 0.000000 & 0.000000 \\
B1 & 7 & 15.000000 & 0.000000 & None & 0.000000 & 0.000000 & 0.000000 & 0.000000 \\
B2 & 1 & 12.000000 & 0.000000 & None & 32.000000 & 0.000000 & 0.000000 & 0.000000 \\
B2 & 2 & 12.000000 & 0.000000 & None & 16.800000 & 0.000000 & 0.000000 & 0.000000 \\
B2 & 3 & 12.000000 & 53.800000 & D2 & 3.120000 & 544.210000 & 50.000000 & 594.210000 \\
B2 & 4 & 12.000000 & 0.000000 & None & 44.610000 & 0.000000 & 0.000000 & 0.000000 \\
B2 & 5 & 12.000000 & 0.000000 & None & 28.150000 & 0.000000 & 0.000000 & 0.000000 \\
B2 & 6 & 12.000000 & 0.000000 & None & 13.330000 & 0.000000 & 0.000000 & 0.000000 \\
B2 & 7 & 12.000000 & 0.000000 & None & 0.000000 & 0.000000 & 0.000000 & 0.000000 \\
\bottomrule
\end{tabular}
\end{table}

\begin{figure}[H]
    \centering
    \includegraphics[width=\textwidth]{img/test_5_1.png}
    \includegraphics[width=\textwidth]{img/test_5_2.png}
    \includegraphics[width=\textwidth]{img/test_5_3.png}
\end{figure}

\textbf{Wynik:} Model poprawnie uwzględnił nadchodzące dostawy w bilansie otwarcia. W pierwszym dniu (i częściowo drugim) system nie generował nowych, zbędnych zamówień, konsumując zapasy z dostaw historycznych (``w drodze''), co pozwoliło uniknąć przepełnienia magazynu i zbędnych kosztów.

\end{document}
